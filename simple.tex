\documentclass[oneside,final,14pt]{extreport}
\usepackage[koi8-r]{inputenc}
%\usepackage[utf8x]{inputenc}
\usepackage[russianb]{babel}
\usepackage{vmargin}
\setpapersize{A4}
\setmarginsrb{2cm}{1.5cm}{1cm}{1.5cm}{0pt}{0mm}{0pt}{13mm}
\usepackage{indentfirst}
\sloppy
\begin{document}
Фольклор~--- явление очень интересное и многоплановое, не
перестающее занимать умы иссследователей. Каких только
способов не изобретают люди, чтобы повеселиться. Вот,
например, неизвестный автор взял два серьезных стихотворения
двух поэтов-классиков и сделал из них своеобразный винегрет:

\bigskip
\noindent Однажды, в студенную зимнюю пору, \\
Сижу за решеткой в темнице сырой. \\
Гляжу поднимается медленно в гору \\
Вскормленный в неволе орел молодой, \\
И, шествуя важно, в спокойствии чинном,\\
Мой грустный товарищ, махая крылом, \\
В больших сапогах, в полушубке овчинном \\
Кровавую пищу клюёт под окном.
\bigskip

Вот такое "<народное творчество>", извольте видеть.
\end{document}

